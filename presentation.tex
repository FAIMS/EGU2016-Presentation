\environment envpresentation


\setlayerframed[footy][frame=off, x=.01\pagewidth, y=.015\pageheight, align=flushleft, width=.75\pagewidth]%
{\color[white]{\setupbodyfont[90pt]\setupinterlinespace[line=0pt]\setupwhitespace[none]Using the FAIMS Mobile App for Field Data Recording}}


\setlayerframed[footy][frame=off, x=.01\pagewidth, y=.1465\pageheight, align=center]%
{\color[white]{\setupbodyfont[28pt]\setupinterlinespace[line=0pt]\setupwhitespace[none]Brian Ballsun-Stanton (Macquarie University), Jens Klump (CSIRO), Shawn Ross (Macquarie University)}}


\definebar[a][color=blue]
\starttext

\setuplist
    [section]
    [alternative=e, color=blue, style={\tfc}, prefix=yes, width={\textwidth}]

{\tfd Quick Navigation:}

\placecontent
\vfill

Abstract:

{\setupbodyfont[14pt]\setupwhitespace[small]Multiple people creating data in the field poses a hard technical problem: our \quote{web 2.0} environment presumes constant connectivity, data \quote{authority} held by centralised servers, and sees mobile devices as tools for presentation rather than tools for origination. A particular design challenge is the remoteness of the sampling locations, hundreds of kilometres away from network access. The alternative, however, is hand collection with a lengthy, error prone, and expensive digitisation process.

This poster will present a field-tested\footnote{See chapter by Sobotkova et. al. in forthcoming book Mobilizing the Past due out at the end of 2016} open-source solution to field data recording. This solution, originally created by a community of archaeologists, needed to accommodate diverse recording methodologies. The community could not agree on standard vocabularies, workflows, attributes, or methodologies, but most agreed that \quote{recording data in the field} was a desireable app to have. As a result, the app is generalised for field data collection; not only can it record a range of data types, but it is deeply customisable.

The NeCTAR funded FAIMS Project, therefore, created an app which allows for arbitrary data collection in the field\footnote{Ross, S., et. al. (2015). Building the bazaar: enhancing archaeological field recording through an open source approach. In Wilson, A. T., Edwards, B. (Eds.). Open Source Archaeology: Ethics and Practice. Walter de Gruyter GmbH Co KG.}. In order to accomplish this ambitious goal, FAIMS relied heavily on OSS projects including: spatialite and gdal (for GIS support), sqlite (for a lightweight key-attribute-value datastore), Javarosa and Beanshell (for UI and scripting), and the entire linux stack plus ruby for a server.

Only by standing on the shoulders of giants, FAIMS was able to make a flexible and highly generalisable field data collection system that CSIRO geoscientists were able to customise to suit most of their completely unanticipated needs\footnote{Reid, N., et. al. (2015) A mobile app for geochemical field data acquisition. Poster session presented at the meeting of AGU Fall Meeting, San Francisco}. While single-task apps (i.e. those commissioned by structural geologists to take strikes and dips) will excel in their domains, other geoscientists (palaeoecologists, palaeontologists, anyone taking samples), likely cannot afford to commission domain- and methodology- specific recording tools for their custom recording needs. FAIMS shows the utility of OSS software development and provides geoscientists a way forward for edge-case field data collection. Moreover, as the data is completely open and exports are scriptable, federation with other data formats is both encouraged and possible.

This poster will describe the internal architecture of the FAIMS app, show how it was used by CSIRO in the field, and display a graph of its OSS heritage. Code for this module is available at: https://github.com/FAIMS/CSIRO-Geochemistry-Sampling. The module can be explored by downloading FAIMS from google play and downloading the *CSIRO Geochemical Sample Collection - Sydney Maps* module from the demo server. A getting started guide is available at: http://faims.edu.au

Source code for this presentation (compiled in \ConTeXt)~is available at: https://github.com/FAIMS/EGU2016-Presentation
}

\section{Developing a multi-device field recording system}
\setupcombinations[distance=10px, width=\textwidth, align=flushleft]
\startcombination[4*2] 
{\externalfigure[noteBook1.jpg][width=.2\textwidth]}{Geosampling data was initially collected via notebook.}
{\externalfigure[noteBook2.jpg][width=.1\textwidth]\externalfigure[noteBook3.jpg][width=.1\textwidth]}{CSIRO Researchers then moved to pre-printed field notebooks for more accurate data entry. }
{\externalfigure[CSIROWireframe.png][width=.2\textwidth]}{Imported into FAIMS Mobile, the CSIRO workflow was modeled in the generalised field recording app.}
{\setupcombinations[distance=5px, width=fit]

\startcombination[2*2]
{\externalfigure[dataSchema.png][width=.1\textwidth]}{Data Schema (XML)}
{\externalfigure[uiLogic.png][width=.1\textwidth]}{Logic (Beanshell)}
{\externalfigure[uiSchema.png][width=.1\textwidth]}{UI Schema (XML)}
\stopcombination}
{These are the primary files which implement a scriptable model, view, and controller field data recording implementation.}
%nextrow
{\externalfigure[app2.png][width=.09\textwidth] \externalfigure[app3.png][width=.09\textwidth]}{After extensive testing, an app was deployed. These are three screens, designated by the wireframe and scripted by the files before.}
{\externalfigure[establishingShot.jpg][width=.2\textwidth]}{Data was collected on multiple tablets in the field. These tablets were offline to save battery.}
{\externalfigure[faimsInTruck.jpg][width=.2\textwidth]}{FAIMS is designed to work completely offline, allowing asynchronous work on multiple tablets with eventual sync. The server, here, was built into a portable UPS in the truck.}
{\externalfigure[exportData.png][width=.2\textwidth]}{After return to base, data exported (via customisable exporter) into shapefiles, a sqlite database, and CSVs. All pictures are renamed to the record they belong to and tagged with their record's metadata.}
\stopcombination



\section{Details of FAIMS}
\setupcombinations[distance=10px, width=\textwidth, align=flushleft]

\startcombination[2*1]
{\externalfigure[dknf.png][width=.5\textwidth]}{
Records are defined in 3 logical tables\footnote. \quote{Rows} are defined in the ArchEntity table, which also holds cruical GIS data. \quote{Columns} are defined by the attribute and Ideal Entity tables. The Attribute table defines what attributes are possible, their names, and their list/export formats. The Ideal Entity table defines which attributes belong to which entity. By defining these tables in DML (Data {\em Manipulation} Language) rather than DDL (Data {\em Definition} Language), the structure of the database remains consistent. This consistent structure allows for significant query reuse and allows us to dynamically script the fields of a workflow {\em after} all the fundamental data interactions of the app have been rewritten.

\vfill
\vfill
{\tfx Image by Geoff Matheson.}

}
{\externalfigure[test.png][width=.45\textwidth]}{ 
The only way an app this complex would be possible would be via the contributions of many open source projects: 
\startitemize[joinedup, packed]
\item JavaRosa which allows us to parse XML into native android elements;
\item NativeCSS which allows us to include some runtime stylings for elements;
\item BeanShell which allows us to dynamically include a java-like scripting language;
\item Spatialite which allows GIS operations inside the database;
\item Sqlite which is a supremely stable single-user database;
\item Ruby, Apache, Linux which allows us to write a sophisticated server running on a completely open source stack.
\stopitemize
}
\stopcombination



\section{CSIRO Workflow}

FAIMS Mobile is a FOSS Software platform (comprising an android client and ruby server on ubuntu) funded by the Australian Research Council designed to provide a means of collecting rich, geospatial, and multi-media field data on multiple tablets with no network connectivity in the middle of nowhere. While originally intended to support archaeologists, FAIMS Mobile provided a sufficiently general field recording framework to allow for geochemical and biological sampling by multiple teams of CSIRO researchers.
\section{Module Details}
\section{Faims internal architecture}
\section{OSS Heritage graph}




\stoptext

